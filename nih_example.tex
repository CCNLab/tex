%\documentclass{article}
\documentclass[12pt]{article}%
% last revision:
\def\mydate{02/11/13 11:42:14 brd}
%\usepackage{scaledfullpage}
\usepackage[dvips]{graphicx}
\usepackage{color}
\usepackage{boxedminipage}
\usepackage{amsfonts}
\usepackage{times}
\usepackage{nih}		% PHS 398 Forms
%\usepackage{nihblank}		% For printing on Blank PHS 398 Forms
%\usepackage{confidential}
%\input{sq}

%Note from brd
\long\def\todo#1{#1}
\def\ICRA{IEEE International Conference on Robotics and Automation (ICRA)}

\long\def\squeezable#1{#1}

%\def\a5{$\alpha_{_5}$

\def\a5{5}

%\def\mycaptionsize{\normalsize}
%\def\mycaptionsize{\small}
%\def\mycaptionsize{\small}
\def\mycaptionsize{\footnotesize}
\def\mycodesize{\footnotesize}
\def\myeqnsize{\small}

\def\sheading#1{{\bf #1:}\ }
\def\sheading#1{\subsubsection{#1}}
%\def\sheading#1{\bigskip {\bf #1.}}

\def\ssheading#1{\noindent {\bf #1.}\ } 

\newtheorem{hypothesis}{Hypothesis}
\long\def\hyp#1{\begin{hypothesis} #1 \end{hypothesis}}

\def\cbk#1{[{\em #1}]}

\def\R{\mathbb{R}}
\def\midv{\mathop{\,|\,}}
\def\Fscr{\mathcal{F}}
\def\Gscr{\mathcal{G}}
\def\Sscr{\mathcal{S}}
\def\set#1{{\{#1\}}}
\def\edge{\!\rightarrow\!}
\def\dedge{\!\leftrightarrow\!}


\def\degree{$^\circ$}
\def\R{\mathbb{R}}
\def\Fscr{\mathcal{F}}
\def\set#1{{\{#1\}}}
\def\edge{\!\rightarrow\!}
\def\dedge{\!\leftrightarrow\!}

\long\def\gobble#1{}
\def\Jigsaw{{\sc Jigsaw}}
\def\ahelix{\ensuremath{\alpha}-helix}
\def\ahelices{\ensuremath{\alpha}-helices}
\def\ahelical{$\alpha$-helical}
\def\bstrand{\ensuremath{\beta}-strand}
\def\bstrands{\ensuremath{\beta}-strands}
\def\bsheet{\ensuremath{\beta}-sheet}
\def\bsheets{\ensuremath{\beta}-sheets}
\def\hone{\ensuremath{^1}\rm{H}}
\def\htwo{$^{2}$H}
\def\cthir{\ensuremath{^{13}}\rm{C}}
\def\nfif{\ensuremath{^{15}}\rm{N}}
\def\hn{\rm{H}\ensuremath{^\mathrm{N}}}
\def\hnone{\textup{H}\ensuremath{^1_\mathrm{N}}}
\def\ca{\rm{C}\ensuremath{^\alpha}}
\def\catwel{\ensuremath{^{12}}\rm{C}\ensuremath{^\alpha}}
\def\ha{\rm{H}\ensuremath{^\alpha}}
\def\cb{\rm{C}\ensuremath{^\beta}}
\def\hb{\rm{H}\ensuremath{^\beta}}
\def\hg{\rm{H}\ensuremath{^\gamma}}
\def\dnn{\ensuremath{d_{\mathrm{NN}}}}
\def\dan{\ensuremath{d_{\alpha \mathrm{N}}}}
\def\jconst{\ensuremath{^{3}\mathrm{J}_{\mathrm{H}^{\mathrm{N}}\mathrm{H}^{\alpha}}}} 
\def\cbfb{CBF-$\beta$}

\newtheorem{defn}{Definition}
\newtheorem{claim}{Claim}

\newenvironment{closeenumerate}{\begin{list}{\arabic{enumi}.}{\topsep=0in\itemsep=0in\parsep=0in\usecounter{enumi}}}{\end{list}}
\def\CR{\hspace{0pt}}           % ``invisible'' space for line break



\begin{document}

\bigskip

\appendix 

%\mydate

\setcounter{page}{20} % or whatever

\noindent{\Large\bf Research Plan}

\medskip 



\section{Specific Aims}

While automation is revolutionizing many aspects of biology, the$\ldots$

\section{Background and Significance}\label{sec:background}

%\vspace*{-0.5\baselineskip}
%\subsection{Introduction}

Modern automated techniques are revolutionizing many aspects of
biology, for example, supporting extremely fast gene sequencing and
massively parallel gene expression testing
(e.g.~\cite{chen99,hartuv99,karp99}).  Protein structure
determination, however, remains a long, hard, and expensive task.
High-throughput structural genomics is required in order to apply
modern techniques such as structure-based drug design on a much larger
scale.  In particular, a key bottleneck in structure determination by
nuclear magnetic resonance (NMR) is the {\em resonance assignment}
problem --- the mapping of spectral peaks to tuples of interacting
atoms in a protein.  For example, spectral peaks in a 3D nuclear
Overhauser enhancement spectroscopy (NOESY) experiment establish
distance restraints on a protein's structure by identifying pairs of
protons interacting through space.  Assignment is also directly useful
in techniques such as structure-activity relation (SAR) by
NMR~\cite{shuker96,hajduk97} and chemical shift mapping~\cite{chen93},
which compare NMR spectra for an isolated protein and a protein-ligand
or protein-protein complex.

This proposal addresses automated assignment and high-throughput
protein structure determination from sparse, unassigned NMR data$\dots$

\section{Preliminary Studies/Progress Report}\label{sec:prelim}

At the heart of our proposal is the {\Jigsaw} algorithm, a novel$\ldots$

\section{Research Design and Methods}\label{sec:design}  

We review the information content of the NMR spectra used by $\ldots$



\section{Research Timeline}
\vspace*{-.1in}
{\small
\begin{center}
\begin{boxedminipage}{3.2in}

\begin{description}

\item[\underline{Year 1}:] Develop novel graph algorithms for $\ldots$

\item[\underline{Year 2}:] Integrate spectral processing $\ldots$

\item[\underline{Year 3}:] Implement and test $\ldots$

\end{description}

\end{boxedminipage}
\hspace*{0.3in}
\begin{boxedminipage}{3.2in}

\begin{description}

\item[\underline{Year 4}:] Algorithms for $\ldots$

\item[\underline{Year 5}:] Based on testing, $\ldots$

\end{description}

\end{boxedminipage}

\end{center}}


%\newpage

\section{Human Subjects}

No human subjects are involved.

\section{Vertebrate Animals}

No vertebrate animals are involved.

\bibliographystyle{plain}
\bibliography{jigsaw,brd,brd-bio,extra,r01}

\section{Consortium/Contractual Arrangements}

None.

%\section{Consultants/Collaborators}

\section{Consultants}

None.
%\todo{}

\eject

%\input{appendix}

\end{document}

%%% Local Variables:
%%% write-file-hooks:   (time-stamp)
%%% time-stamp-active:  t
%%% time-stamp-start:   "\\\\def\\\\mydate{"
%%% time-stamp-end:     "}"
%%% time-stamp-line-limit: 20
%%% End:


