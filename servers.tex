%%%%%%%%%%%%%%%%%%%%%%%%%%%%%%%%%%%%%%%%%%%%%%%%%%%%%%%%%%%%%%%%%%%%%%%
% \magnification=\magstep1
\font\rm=cmssq8
\font\it=cmssqi8
\font\bf=cmssdc10 at10.95truept
\font\bx=cmbx10 scaled \magstep2
\font\bd=cmti10 scaled \magstep2
\font\br=cmr12 at13.14truept
\font\bq=cmssbx10
\font\tt=cmtt9
\newcount\ttflag
\def\tx{\tt\ttflag=1 }
\font\st=cmsl9
\font\trm=cmssq8 at8truept
\font\bt=cmtt10 scaled \magstep1
\rm\baselineskip=15.768truept %ie 14.4pt x \magstephalf
\parskip=\medskipamount \parindent=0pt
\newcount\verno \verno=2
\newtoks\date \date={May 1990}
\footline={{\trm v\number\verno\ (\the\date)}
\hfil{\trm Network Sources of \TeX ware}\hfil{\trm p\folio}}
\long\def\sec#1 #2{\bigbreak{\bd#1}\nobreak\smallskip#2}%
\long\def\ser#1 #2{\bigbreak{\bt#1}\quad{\br (#2)}}
\def\nl{\hfil\break}
\def\bull #1 {\par\parindent=2pc\item{#1}\let\par=\endgraf
    \def\par{\endgraf\parindent=0pt\let\par=\endgraf}}
\def\doublebull #1 {\par\parindent=2pc\itemitem{#1}\let\par=\endgraf
    \def\par{\endgraf\parindent=0pt\let\par=\endgraf}}
\def\und#1{$\underline{\hbox{{\tx#1}}}$}
\def\LaTeX{{\rm L\kern-.36em\raise.3ex\hbox{\trm A}\kern-.15em
     T\kern-.1667em\lower.7ex\hbox{E}\kern-.125emX}}
\catcode`\<=\active \catcode`\>=\active
\def<{\ifnum\ttflag=1 \char'074 \else\ifmmode<\else$\langle$\fi\fi}
\def>{\ifnum\ttflag=1 \char'076 \else\ifmmode>\else$\rangle$\fi\fi}
%
% Delete up as far as here if you are using this as an ASCII file only
% (ie without TeX).
%
\centerline{{\bx
                        NETWORK SOURCES OF \TeX WARE}}
\bigskip
%
\centerline{{\bq
               How to get public domain and shareable software}}
\centerline{{\bq
                       for the \TeX\ typesetting system}}
\centerline{{\bq
               from the international computer network servers}}
\bigskip
%
\centerline{{\rm
                                 Peter Flynn}}
\centerline{{\it
              Computer Centre, University College, Cork, Ireland}}
\centerline{         <{\tx cbts8001@iruccvax.ucc.ie}>}
\bigskip
%
\bigskip
\sec{Introduction}

     On the international computer networks there are various locations
     where generous individuals and institutions have placed freely
     accessible software, including a considerable amount for the \TeX\
     typesetting system. These computers (or rather, the programs which
     let you access the information) are known as `servers', and anyone
     with access to electronic mail (E-mail) can request copies of files
     from these servers to be sent to them electronically.

     A server consists of disk space on a computer connected to a
     network, governed by a program capable of receiving instructions
     from elsewhere on the network and responding to them. Some servers
     use programs specifically dedicated to the single task of running
     that particular server (eg LISTSERV); others use more
     general-purpose software (eg FTP). A computer offering a server
     service is called a `host'.

     Users of the network can thus send instructions to the host program
     via the network, and expect the host to act upon them. These
     instructions are called `commands', and a command to a server is
     typically a means of telling the host to send you a copy of a
     specific file from the server disk. A server will respond by
     sending the file or (if the filename was wrongly given, for
     example) by returning a message explaining the error.

     The \TeX\ software available ranges from simple routines to perform
     individual formatting tasks, through a wide variety of macro
     packages for more complex requirements, right up to complete
     implementations of the whole \TeX\ system. Also available are many
     font files and printer handlers (drivers), and an amount of unusual
     or experimental typographic facilities. Several people and
     organisations also operate a mail-order service for those users
     without network access.

{\bf Finally, please note that some software on the networks is in the
     public domain (it can be used by anyone without charge), but some
     is shareware (it can be tried out without charge, but must be
     registered and paid for if you continue to use it---usually only a
     small sum). Please do not continue to use shareware without paying
     your contribution: it is both dishonest and unfair.}

\sec{Principles of network usage}

     If you have not used E-mail or other networking services before,
     you should ask your computer centre or network operator for
     documentation and training. Although most systems have software
     which is fairly simple to use, getting the best out of network
     access means being reasonably familiar with the commands and
     facilities your system provides. There is no point in having
     wonderful access to network software services if you spend most of
     your time trying to remember which menu option to pick or which
     function keys you press.

     You can gain access to the servers by various methods, depending on
     what network you yourself are attached to, what networking
     facilities it provides, and what access services the server is set
     to accept. You should check with your local computer centre or
     network operator if you do not know what facilities are currently
     provided or how to use them.

     As mentioned at the outset, the only facility common to all
     networks is E-mail, and this now functions with reasonable
     reliability across most networks. Other facilities available within
     some networks include file transfer (FTP, in several flavours---see
     below); interactive login (dialling into a remote computer); and
     interactive messaging (sending single commands in real time
     without logging in). Not all of these other facilities are
     available on all networks, and apart from E-mail, they will not
     work at all if the server is on another network using different
     software to your own.

     Whichever method you choose, the principle remains the same: you
     send commands to the server address. For example, you might send
     the command to have a file transmitted back to you, followed by the
     name of the file you want. Provided you have typed the address and
     command(s) correctly, the result will be the arrival on your
     computer of the file you ordered.

\sec{Handling files you have retrieved}

     Files you order may arrive by E-mail or by file transfer, depending
     on the network you are connected to and how you ordered them.
     Generally speaking, a file is returned to you by the same mechanism
     by which you requested it, so if you used E-mail to ask for a file,
     you get it back by E-mail; if you used an FTP request, you get it
     back by FTP. {\bf It is important to appreciate that E-mail in its
     current state is normally a printable-character-only medium and so
     can only be used for plain text files, or for other files which
     have been encoded into printable characters only (see below).}

     Plain text files cause no problems in 99\% of cases. However, in
     the case of files being transferred between two different networks
     where the `gateway' machine (the computer performing the
     interconnection) has an unusual, specialist or ideosyncratic
     character-conversion table, a few characters get mistranslated. The
     most common mistranslation is to send you tildes ({\tx\~{}}) in
     place of caret marks ({\tx\^{}}), and ASCII decimal character codes
     197 and 185 (box-drawing characters on the IBM PC) in place of open
     and close curly-braces. This can usually be fixed with a good text
     editor. If you have problems in receiving or deciphering files you
     have ordered, contact your computer centre or network operator, but
     be prepared to hold discussions with someone from the gateway
     through which the file or mail passed. If an expected plain text
     file does not process correctly, this mistranlation is one of the
     first things to suspect.

     If the file you are ordering is not a plain text file (for example,
     if it is a `binary' file like an executable program, a
     wordprocessing file, a font file or a compressed archive of files),
     it cannot usually be sent in its raw state by E-mail, particularly
     between two machines of different makes, or between different
     networks.

     (Remember also that an executable program for one operating system
     will not work on a different one: make sure you request such
     programs for the right operating system!)

     Many servers are able to overcome the problem of sending binary
     files by E-mail or across network boundaries by encoding such files
     into a new file made up of printable characters only, which they of
     course can then send by E-mail. This is normally something you can
     specify when you order a file, but some servers do it automatically
     if you use E-mail to order the file. You can recognize a file coded
     like this because it contains only printable characters, and
     usually the lines are all the same length, and the first line of
     the file will say something like {\tx begin} or
     {\tx FfIiLlEeSsTtAaRrTt}. The most commonly-used method (for IBM
     and DEC mainframes, most minis, UNIX and PCs) is called UUencoding,
     and you will require the UUDECODE program already to be on your
     machine in order to decode such files if you order them.

     The catch is, because this program is itself a binary executable
     file, you cannot receive it in UUencoded form by E-mail unless you
     already have a copy with which to UUdecode it! To overcome this
     chicken-and-egg situation, you must either request it in source
     code form and compile it yourself, or obtain an executable copy on
     disk or from some other source, such as dial-up download from a
     bulletin board system. This technique is known as `bootstrapping'
     yourself (not to be confused with `booting' your computer). Your
     computer centre or network operator should also be able to supply a
     copy of UUDECODE for your system. A known location for the source
     code is given at the end of this document. The UU programs are
     believed to be in the public domain.

     Other coding systems in use are XXENCODE and XXDECODE (a more
     recent and robust version of UU); BIN2HEX, which converts the bytes
     of a binary file to pairs of hexadecimal characters (and HEX2BIN
     which converts them back again); and BOO and DEBOO (short for
     `bootstrap') which is used extensively for distributing the Kermit
     communications program, and also for some software sourced from
     within the UK. Apple Macintoshes use a version of BINHEX, but
     differently implemented because of the Mac's twin-forked filing
     system: again, you need to get a copy of it from someone on disk
     before you can start.

     All these encoding systems get over the problem of transferring
     binary files over E-mail, but they all suffer from the disadvantage
     that the encoding increases the file size, sometimes quite
     substantially. To partially overcome this, files, even text files,
     are sometimes compressed with a compression program before being
     encoded for transmission.

     Collections of related files, especially for PCs, Macs and UNIX,
     are often also compressed into a single file for ease of
     transmission. This is called `compression archiving', and is the
     most popular method of compacting files. The resulting single
     filename is easily recognisable by usually having a filetype or
     extension of {\tx .arc}. A similar mechanism for UNIX is called
     TAR, and for Macintoshes it is called STUFFIT. To unpack the file
     once you have received it, you need the relevant de-archiving
     decompression program: there are several available under various
     names from most bulletin boards and servers. For MS-DOS PCs, the
     original archive/de-archive programs were ARC and ARCE from System
     Enhancement Asociates, but the current leader is a piece of
     shareware, PKPAK/PKUNPAK (replacing the older PKARC/PKXARC, and
     there is also a newer one called PKZIP/PKUNZIP). ARC also exists
     for VAX/VMS and for VM/CMS. If you are using UNIX, the TAR programs
     should already be on your system. The Mac program UNSTUFFIT is
     freely distributable and should be available from your dealer (the
     STUFFIT program to create archives is a commercial product,
     however).

\sec{Known \TeX\ servers as at \the\date}

     You should be aware that there are many more servers handling a
     wide range of non-\TeX\ software, both text and data; and that
     there are many other commands as well---only the most important are
     given here. As a general principle, sending the single word
     `{\tx HELP}' (without quotes) to an address claiming to be a server
     is as good a way as any of testing its likely usefulness!

     Please inform the author of any changes, additions, deletions and
     errors.

\ser{listserv@dhdurz1.bitnet} {Heidelberg University Computer Centre}

{\bf Access by}: E-mail, RSCS interactive message, RSCS FTP

{\bf Commands}:\nl
{\tx HELP}\quad sends you back a help file describing LISTSERV.\nl
{\tx SENDME} {\st filename filetype} $[${\st (tag\/}$]$\quad sends you
     the specified file. The file specs are all in IBM VM/CMS format,
     consisting of a filename and a filetype separated by a space, but
     LISTSERV will accept a filename and filetype separated by a dot
     instead of a space. {\tx SEND} and {\tx GET} are synonyms for
     {\tx SENDME}. If you are ordering a non-printable (program or
     archive) file, you can follow the filetype with the optional tag
     `{\tx (UUE}' in order to have the file sent in UUencoded form. Note
     there is no closing parenthesis on a LISTSERV {\tx SENDME} tag.\nl
{\tx INDEX}\quad sends back a list of files. More detailed lists are
     held in files with the filetype {\tx FILELIST}.

{\bf Examples}:\nl
{\tx sendme listserv filelist}\quad will send you the list of
     {\tx FILELIST} files on the server from which you can identify
     further lists.\nl
{\tx send drivers filelist}\quad will send you the list of \TeX\ print
     driver files which can be retrieved.\nl
{\tx get mtex arc (uue}\quad would request a UUencoded copy of the
     {\tx mtex} archive file.\nl
{\tx help}\quad would request the help file from LISTSERV.

{\bf Notes}: If you send your request by mail, the response comes back
     by mail. If you send it by interactive message or by RSCS FTP (the
     {\tx TELL} or {\tx SENDFILE} commands on IBM VM/CMS under RSCS; the
     {\tx SEND/REM} or {\tx SEND/FILE} commands on DEC VAX/VMS under
     JNET), the response comes back by file transfer. There are many
     other {\tx LISTSERV}s around the world which may also have
     unreported \TeX\ file collections. Known ones are listed below.

{\tx LISTSERV} also handles EARN/BITNET mailing lists, including the
     \TeX hax Bulletin. To subscribe to a mailing list, send an
     interactive message or a one-line E-mail to any {\tx LISTSERV}
     saying:\nl
{\tx SUBSCRIBE} {\st LISTNAME~your-real-name}\quad eg\nl
{\tx SUB TEXHAX Mary Jones}\nl
     You will then start to receive the digest of mailings from other
     contributors, and you can send your own contributions to the
     address of the mailing list ({\it not\/} to {\tx LISTSERV}),
     for example, the \TeX hax Bulletin editorial address is
     {\tx texhax@cs.washington.edu}.\nl
     It is important to understand that LISTSERV subscription requests
     (and un-subscription requests, which are done with the command {\tx
     unsub}~{\st listname}) must be sent to a {\tx LISTSERV} and {\it
     not\/} to the address of the mailing list itself. \nl
     Intending subscribers on the DARPA Internet should send a message
     to the manually-operated address
     <{\tx texhax-request@cs.washington.edu}> instead.

     The Heidelberg server includes the Beebe driver collection and the
     \LaTeX\ style file collection. It also hosts the {\tx tex-euro}
     list, for discussions of specifically European \TeX\ problems.

\ser{listserv@dearn} {Universit\"at Bonn, Germany}

     Holds subscriptions for lists {\tx\TeX\_D-L} (German-language
     \TeX\ discussion) and {\tx\TeX\_D-PC} (German-language \TeX-on-PCs
     discussion).

\ser{listserv@hearn} {Katholiecke Universiteit Nijmegen}

{\tx tex-nl filelist} contains a large quantity of Dutch \TeX\ material
     and is reported to be starting an Atari~ST archive soon.

\ser{listserv@frulm11} {\'Ecole Normale Sup\'erieure, Paris}

     Handles subscriptions to GUT, the French-language \TeX\ discussion
     and communication channel for GUTenberg, the French \TeX\ Users
     Group. Subscribe {\tx GUT}

\ser{listserv@tamvm1.bitnet} {\TeX as A\&M}

     Contains a very large repository of \TeX\ material.

\ser{listserv@tcsvm.bitnet} {Tulane University}

     Has back issues of \TeX Mag in files {\tx TEXMAG~VvNn} where
     <{\tx Vv}> is the volume number and <{\tx Nn}> is the issue number.

\ser{listserv@ubvm.bitnet} {University of New York at Buffalo}

     Files related to the Russian \TeX\ project are listed in {\tx
     RUSTEX-L~FILELIST}

\ser{listserv@uicvm.bitnet} {University of Illinois, Chicago}

     This server runs the {\tx tex-ed} mailing list, formed at the 10th
     TUG Conference, to handle educational matters relating to \TeX. It
     is the source for Michael Doob's {\it Gentle Introduction to
     \TeX}.\nl
     It also hosts the distribution of \TeX Mag, an
     independently-published electronic magazine sporadically
     bi-monthly: subscribe {\tx TEXMAG-L} (CDNnet users please send
     your request to the manually-operated address
     <{\tx list-request@ubc.csnet}> and JANET users to <{\tx
     abbottp@aston.ac.uk}>

\ser{texserver@tex.aston.ac.uk} {Aston University, Birmingham}

{\bf Access by}: E-mail, Coloured Book FTP, Post

{\bf E-mail usage}: All requests to the Aston mail-server should be
     preceded by a line starting with three dashes (`{\tx---}'). This
     will normally be the first line of the text body of your mail
     message. Only one command will be processed in each mail message.
     The next non-blank line following the three dashes should contain
     your return address from Aston (see below for examples). Your
     return address {\bf must} be given in UK (JANET) format. The
     following line should then contain the command to the mail-server.
     An example request might therefore look like this:\nl
{\tx ---}\nl
{\tx cbts8001\%iruccvax.bitnet@earn-relay}\nl
{\tx whereis tex.exe}\nl
     Lines before the triple dash are ignored, as is all text after the
     first command.

{\bf Commands}:\nl
{\tx HELP}\quad sends you back a help file describing TEXSERVER. Help in
     languages other than English can be obtained by typing {\tx
     HELP/}{\st language}, eg {\tx HELP/FRANCAIS} (if there is no help
     for your requested language, you will be sent the English
     version).\nl
{\tx DIRECTORY} $[${\st directory-specification}$]$\quad sends you a
     list of the files in that directory. The directory specification
     must be in VAX/VMS syntax including the square brackets: see the
     help file for details. If no directory specification is supplied,
     you will be sent back a list of the files in the top-level
     directory of the archive, {\tx [tex-archive]} \nl
{\tx WHEREIS} {\st filename}\quad sends you a message containing the
     location in the archive of the requested file. If no filename is
     supplied, you will be sent a listing of all files in the archive
     whose names start with `{\tx 00}' (two zeroes), conventionally used
     for descriptions.\nl
{\tx SEARCH} {\st filespec~search-string}\quad searches the specified
     file(s) for the given string and returns the fully-qualified file
     specification. The search is case-independent.\nl
{\tx FILES}\quad followed by a list of the files to be returned,
     specified one per line on succeeding lines. Wildcards are not
     supported. Each requested file is normally returned in a separate
     mail message.

{\bf Notes:} Users in the UK should express their return address in
     the form:\nl
{\tx user@UK.AC.site.machine} (Non-academic users replace the
     `{\tx UK.AC}' with `{\tx UK.CO}')\nl
     EARN/BITNET users must express their return address in the form:\nl
{\tx user\%nodename.BITNET@EARN-RELAY} in order for JANET to be able to
     send replies out through the EARN gateway correctly.\nl
     Users on other networks (eg, {\tx .EDU}, {\tx .COM} etc) should
     express their return address in a similar form to the EARN/BITNET
     one: {\tx user\%machine.site.EDU@EARN-RELAY}, since the
     NSFNET-RELAY gateway is not available for UK-to-US traffic.\nl
     Note that the order of specifying domains {\bf is} important to the
     EARN gateway. The syntax is similar for other networks accessed via
     the this gateway. Other users should try EARN in the first
     instance. If that fails, consult a local networking guru. If that
     fails, mail the archive maintainer, Peter Abbott <{\tx
     abbottp@aston.ac.uk}>, who should be able to put you in touch with
     someone who can help.

{\bf Using Coloured Book FTP (NIFTP):} Use your local {\tx TRANSFER}
     command (part of the Coloured Book suite of XXX implementations)
     with the userID `{\tx public}' and the password `{\tx public}'.
     Give the fully-qualified nodename, directory and filename as the
     remote filename, and whatever you want as your local filename
     (where to put it when it arrives). A file transfer gateway between
     JANET and EARN/BITNET will be introduced experimentally during
     1990.

{\bf Return addresses}: As a temporary facility (pending rewriting of
     the mailer to obviate the need for users to quote their own
     addresses) Brian $\{$Hamilton Kelly$\}$ has provided a remote
     ``identification'' service at Aston. Send E-mail to <{\tx
     rmcs\_tex@kirk.aston.ac.uk}> with the subject consisting of the
     words `{\tx Where~Am~I}' (the case doesn't matter, and you can have
     as much or as little white space between the words [including
     none]), but there must not be any leading or trailing space, nor
     any question mark. You should receive a reply telling you the
     address to `plug into' your E-mail request to TEXSERVER.

{\bf Examples of file specifications}:\nl
     VAX/VMS directory and filename format is tree-structured: a valid
     fully-qualified name therefore looks like\nl
{\tx [TEX-ARCHIVE.directory.subdirectory]filename.type;version}\quad
     eg\nl
{\tx [TEX-ARCHIVE.msdos.tex]sb08tex.arc;2}\nl
     A remote filename for FTP might look like\nl
{\tx UK.AC.ASTON.TEX::[TEX-ARCHIVE.digests.texhax89]tex89.114;1}

     Aston also handles subscriptions for UK\TeX, a weekly digest along
     the lines of \TeX HaX.
     Requests to <{\tx info-tex-request@aston.ac.uk}> and submissions to
     <{\tx info-tex@aston.ac.uk}>, please.\nl
     There is a <{\tx tex-unmoderated@aston.ac.uk}> which can be used to
     get urgent help on matters which cannot wait the next issue of one
     of the moderated digests. Please do not misuse this service.

{\bf Post}: You can send snailmail to Peter Abbott, Computing Service,
     Aston University, Aston Triangle, Birmingham B4 7ET, England,
     enclosing blank formatted media: floppy disks or magnetic tape.

\ser{[archive-server@]sun.soe.clarkson.edu} {Clarkson University}

     This machine is one of the principal repositories of \TeX ware.

{\bf Access by}: Internet FTP, Mail

{\bf Commands}:  These may vary depending on the implementation of
     Internet FTP. You should ask your System Administrator for
     details.\nl
{\tx ftp}\quad starts an FTP session\nl
{\tx open} {\st machine.site.domain}\quad opens an FTP call to a
     remote machine. Some implementations let you type the nodename
     directly after the {\tx ftp} command. The next command may not be
     needed, depending on how the remote machine is set up.\nl
{\tx user anonymous guest}\nl
{\tx cd} {\st directoryname}\quad connects you with the specified
     directory.\nl
{\tx ls} $[${\st filespec}$]$\quad lists the contents of the current
     directory.\nl
{\tx type} {\st filename}\quad types out the specified file on your
     screen.\nl
{\tx tenex}\quad switches to 8--bit byte-stream mode for getting binary
     files.\nl
{\tx get} {\st filename}\quad retrieves the specified file into your
     local current directory.\nl
{\tx mget} {\st wildcard-filenames}\quad retrieves multiple files
     matching the wildcards ({\tx *} and {\tx ?}).\nl
{\tx ascii}\quad switches back to ASCII mode for text files.

{\bf Example}: The commands typed by the user are \und{underlined}\nl
{\tx \$} \und{ftp sun.soe.clarkson.edu}\nl
{\tx Connected to sun.soe.clarkson.edu}\nl
{\tx 220 SUN.SOE.CLARKSON.EDU Server Process (52)-5 at Tue 2-Jan-90}\nl
{\tx 331 ANONYMOUS user ok, sent real ident as password}\nl
{\tx 230 User ANONYMOUS logged in at Tues 2-Jan-90 14:14-XXX, job 2}\nl
{\tx >} \und{cd /tex/binaries}\nl
{\tx >} \und{dir latex.*}\nl
     (Listing appears on screen)\nl
{\tx >} \und{tenex}\nl
{\tx 200 Type L bytesize 8 ok.}\nl
{\tx >} \und{get latex.exe}\nl
     (File is downloaded)\nl
{\tx >} \und{quit}\nl
{\tx 221 QUIT command received. Goodbye.}\nl
{\tx \$}

     Directory {\tx pub/texmag} holds back issues of \TeX Mag in files
     {\tx texmag.v.nn} where <{\tx v}> is the volume number and
     <{\tx nn}> is the issue number. \nl
     Directory {\tx pub/texhax} holds back issues of \TeX HaX in files
     {\tx texhax.yy.nnn} where <{\tx yy}> is the year number and
     <{\tx nnn}> is the issue number. \nl
     Directory {\tx pub/uktex} holds back issues of UK\TeX\ in files
     {\tx uktex.yy.nnn} where <{\tx yy}> is the year number and
     <{\tx nnn}> is the issue number. \nl
     Directory {\tx pub/latex-style} holds master copies of the \LaTeX\
     style files.

{\bf Notes}:  This server is also accessible by mail: place your
     sequence of FTP commands in a mail message to the address above,
     and make the first command {\tx path} followed by your network
     address in a form in which an Internet machine will understand it.
     Internet sites are also FTP-accessible to BITNET nodes via the
     server at <{\tx BITFTP@PUCC}> (see below).

\ser{archive-server@wsmr-simtel20.army.mil} {The SIMTEL--20 server}

     SIMTEL--20 is a large file server on the Internet and operates in
     the same way as shown above for the Clarkson server. The \TeX ware
     is mainly for PCs and is held in directory {\tx /msdos.tex}\nl
     Non-Internet users should use the TRICKLE or BITFTP servers
     detailed below to access these files.

     The following Internet sites also have \TeX-related material:

{\tx argon.rti.org} (128.109.139.64) \TeX\ Previewer for VMS\nl
{\tx b.scs.uiuc.edu} (128.174.90.2) \LaTeX\nl
{\tx bobcat.csc.wsu.edu} (134.121.1.1) Dean Guenther's IPA fonts,
     CG8600 driver and \TeX T1 style file (documentation chargeable from
     Computing Service Center, Washington State University, Pullman WA
     99164--1220)\nl
{\tx cayuga.cs.rochester.edu} (192.5.53.209) Xfig, \LaTeX\ styles, Jove,
     NL-KR mail list\nl
{\tx crocus.waterloo.edu} (129.97.128.6) STEVIE (vi-clone), \TeX, more\nl
{\tx cs.washington.edu} (128.95.1.4) \TeX, \TeX hax, netinfo\nl
{\tx ctrsci.utah.edu} (128.110.192.4) \TeX\ fonts, make\nl
{\tx duke.cs.duke.edu} (128.109.140.1) gnutex, others\nl
{\tx freja.diku.dk} (129.142.96.1) GNU, X11R3, \TeX, nn newsreader, rfcs,
     misc\nl
{\tx gatech.edu} (128.61.1.1) GNU, rfc, \TeX\nl
{\tx gpu.utcs.toronto.edu} (128.100.100.1) \TeX, C++, Ksh, Unixgames,
     etc. (lots)\nl
{\tx hemuli.atk.btt.fi} (130.188.52.2) bsd progs for hp-ux, tex2ps\nl
{\tx hydra.helsinki.fi} (128.214.4.29) misc, \TeX, X, comp.sources.misc,
     sun, uni\nl
{\tx jpl-mil.jpl.nasa.gov} (128.149.1.101) \TeX, Mac, Gnu, Xv11R{2, 3}\nl
{\tx june.cs.washington.edu} (128.95.1.4) \TeX hax, dviapollo, SmallTalk,
     web2c, gaat\nl
{\tx labrea.stanford.edu} (36.8.0.47) dvips, paranoia, \TeX, lots, X\nl
{\tx linc.cis.upenn.edu} (128.91.2.8) psfig for ditroff, \TeX\nl
{\tx ncar.ucar.edu} (128.117.64.4) maps, bsd, internet, Mac \TeX,
     resolve\nl
{\tx njitgw.njit.edu} (128.235.1.2) Mac, Sun, \TeX\nl
{\tx purdue.edu} (128.102.1) bibtex, dvi, ethics\nl
{\tx research.att.com} (192.20.225.1) \TeX, gcc, ghostscript\nl
{\tx science.utah.edu} (128.110.192.2) \TeX\ things, Hershey (tenex),
     \TeX Mag back issues (file {\tx bbd:texmag.txt}) and \TeX HaX back
     issues (file {\tx bbd:texhax.txt})\nl
{\tx score.stanford.edu} (36.8.0.46) \TeX Hax, Atari, APL metafont
     (tenex)\nl
{\tx sun.soe.clarkson.edu} (128.153.12.3) Packet Driver, X11 fonts, \TeX,
     PCIP, Free\nl
{\tx titan.rice.edu} (128.42.1.30) sun-spots, amiga ispell,
     pc-bibtex.tar\nl
{\tx uicsrd.csrd.uiuc.edu} (128.174.132.2) BibTeX, CommonTeX\nl
{\tx venus.ycc.yale.edu} (192.26.88.4) SBTeX\nl
{\tx walther.cso.uiuc.edu} (128.174.5.20) \TeX, tib, ncar, dvi2ps, gif,
     texx2.7, amiga\nl
{\tx wuarchive.wustl.edu} (128.252.135.4) password: guest, mirrors
     simtel20 (lots), \TeX, Mac, X, GNU, GIF, Tcp-Ip

\ser{bitftp@pucc.bitnet} {Princeton University}

     BITFTP is a mechanism for those without direct access to the
     Internet to request files by FTP from Internet servers (like {\tx
     archive-server@sun.soe.clarkson.edu})

{\bf Access by}:  Mail, RSCS FTP

{\bf Commands}:  The body of your mail message or file should contain
     the sequence of Internet FTP commands you would have used, one per
     line, just as if you had been doing a direct Internet FTP
     connection yourself.

\ser{LaTeX-help@sumex-aim.stanford.edu} {\LaTeX\ helpdesk}

     Your mail will be forwarded to a member of the volunteer corps in
     round-robin rotation.

\ser{fisica@39003.span} {SPAN/DECNET archive}

     There is a \TeX\ archive on SPAN run by Max Calvani and Marisa
     Luvisetto. It is not a server, just an archive: details from
     <{\tx fisica@astrpd.infn.it}> or by sending a SPAN mail message
     to the address on SPAN <{\tx 39003::fisica}>

\ser{trickle@trearn.bitnet} {The TRICKLE server at Ege University,
     \.Izm\i r}

     TRICKLE is a cache mechanism for EARN to allow users on that
     network to request files from SIMTEL--20. TRICKLE runs at various
     sites on EARN (see below).

{\bf Access by}:  Mail, RSCS FTP, Interactive message

{\bf Commands}:\nl
{\tx /HELP}\quad sends you a help file.\nl
{\tx /PDDIR} {\st directoryname}\quad lists the names of files in that
     directory.\nl
{\tx /PDGET <}{\st directory}{\tx>}{\st filename} $[${\st (tag\/}$]$
     \quad sends the specified file. The tag can be {\tx (uue}, {\tx
     (xxe}, {\tx (ebc80} or {\tx (ebc32}, to determine how you want the
     file returned and in what format.

{\bf Example}:\nl
{\tx /pdget <msdos.tex>pcwritex.arc (uue}\quad will order the given file
     from directory {\tx <msdos.tex>} in UUencoded form.\nl
{\tx /pddir <msdos.tex>}\quad will send the list of all files in the
     {\tx <msdos.tex>} directory.

{\bf Notes}: TRICKLE runs at the following other EARN nodes as well:
     $$\vbox{\halign{{\tx#}\quad\hfil&\quad{\it#}\hfil           \cr
     awiwuw11  &  Wirtschaftsuniversit\"at Wien                  \cr
     banufs11  &  Univ.\ Faculteiten Sint-Ignatius te Antwerpen  \cr
     db0fub011 &  Freie Universit\"at Berlin                     \cr
     dktc11    &  Copenhagen Technical College                   \cr
     dtuzdv1   &  Universit\"at T\"ubingen                       \cr
     eb0ub011  &  Universidad de Barcelona                       \cr
     imipoli   &  Politecnico di Milano                          \cr
     taunivm   &  Tel Aviv University                            \cr}}$$
     They are all peered, so you should use the one which is logically
     nearest to your node (measured in network hops).

\ser{jonradel@bogey.princeton.edu} {Jon Radel's Repository}

     Jon Radel offers a mail-order service for those users with no
     access to networking.

{\bf Access by}:  Post

{\bf Commands}:  I quote from his message in the TeXhax bulletin 1989
     No.~13:\nl
{\tx Date: Tue, 7 Feb 89 03:41:23 EST }\nl
{\tx From: jonradel@bogey.Princeton.EDU (Jon Radel) }\nl
{\tx Subject: For those who don't have access to TeX for PCs on the
              net...}\nl
{\tx Keywords: general, TeX, PCs }

   ``Time to introduce myself again. As a service for people who do not
     have decent access to \TeX\ to PC material on the net, I distribute
     much of that material on floppies for a handling charge.  That
     includes the 75 font, 5 magstep collection for a couple of the
     more popular printers, two versions of \TeX, and a variety of
     smaller items.

   ``For various reasons, I do all my dealings on this matter by `snail'
     mail, so you have to send me a self-addressed envelope to get the
     list of material that I have.  45 cents postage inside the USA,  4
     International Reply Coupons or US\$1.60 for airmail elsewhere, half
     that for surface (and Canada/Mexico, where surface is air as far as
     the USPS is concerned).''

{\bf Example}:\nl
     Jon Radel, \nl
     PO Box 2276, \nl
     Reston, \nl
     VA 22090, \nl
     USA

{\bf Notes}:  The \TeX\ community's thanks are due to Jon for his
     provision of this service.

\ser{tex/listings@bytecosy.tower.bix.us} {BIX, the BYTE magazine Information
     Exchange}

     Application has been made to BYTE magazine to start a \TeX\
     conference and listings area in their BIX online conferencing
     and filestore system. No start date has been set for this service
     yet.

{\bf Access by}:  Interactive login via an ordinary X.29 (packet-switched)
     call to [0]310690157800 or using a standard modem ({\tx 8,n,1}) to
     +1~617~861~9767 (BELL tones for 300 and 1200 baud, BELL or CCITT
     for 2400 and up). Press the Enter or Return key and at the login
     prompt type {\tx bix}~. When asked for your name, type {\tx
     bix.flatfee} and you will automatically be taken through the
     new-user signup routine. Files can be downloaded through your modem
     or X.25 connection.

{\bf Commands}:\nl
{\tx join tex}\quad joins you to the {\tx TeX} conference.\nl
{\tx topic listings}\quad joins you to the {\tx listings} area for \TeX\
     files. \nl
{\tx receive} {\st filename}\quad starts downloading to your workstation
     the file you specify. Immediately after pressing the Return or
     Enter key for this command, you need to instruct your workstation
     to receive a file using the XMODEM protocol (by default---other
     protocols are available, such as KERMIT).\nl
{\tx option receive kermit}\quad instructs BIX to use the KERMIT
     protocol when it sends you a file. \nl
{\tx quit}\quad leaves the listings area.\nl
{\tx bye}\quad signs you off back to your PAD or modem link.

{\bf Example}: \nl
{\tx join tex }\nl
{\tx topic listings}\nl
{\tx opt rec ymodem}\nl
{\tx receive dostex.arc}\nl
{\tx quit}\nl
{\tx bye }

{\bf Notes}:  You can only use BIX for downloading files if you are
     equipped with a computer running terminal emulation software which
     includes file-download protocols such as KERMIT, XMODEM, YMODEM or
     similar.

\ser{CIX} {Compulink Information eXchange}

     Like BIX, but UK-based, and without network access (it reputedly
     has USENET mail, but no address is available). CIX has a
     substantial \TeX\ conference, with many files for downloading by
     those with no network access (+44 1 399 5252).

\ser{Channel 1} {Boston, USA}

     A bulletin board with a \TeX\ area since 1987. (+1 617 354 8873)

\bigskip
\hrule%-----------------------------------------------------------------
\bigskip

     Source code for the {\tx uudecode} program is known to reside on
     the SIMTEL-20 server in the following files. They may be ordered
     from TRICKLE in the manner detailed above.

{\tx <CPM.STARTER-KIT>UUDECODE.PAS}\nl
{\tx <MSDOS.STARTER>UUDECODE.BAS}\nl
{\tx <MSDOS.STARTER>UUDECODE.COM}\nl
{\tx <MSDOS.STARTER>UUDECODE.EXE}\nl
{\tx <MSDOS.STARTER>UUDECODE.C}\nl
{\tx <MSDOS.STARTER>UUENCODE.UUE}\nl
{\tx <MSDOS.STARTER>XXDECODE.TXT}\nl
{\tx <UNIX-C.MAIL>UUENCODE-UUDECODE.TAR-Z}

{\tx <MISC.VAXVMS>VMSDECOD.EXE} is an executable: the source is not available.

\bigskip
\hrule%-----------------------------------------------------------------
\bigskip

     Users of commercial mailing and messaging systems will need access
     to the academic and research networks in order to use these
     servers. This is currently available through an organisation called
     DASnet. With a subscription to DASnet (and a small charge per
     1,000 characters either direction) you can send and receive mail
     from your local commercial mailbox to and from the academic and
     research networks. The format of address depends a little on your
     host system, but for an example I quote a mail from their
     coordinator:

{\tx From: IN\%"AnnaB@11.DAS.NET"}\nl
{\tx To: cbts8001@IRUCCVAX.UCC.IE }\nl
{\tx Subj: DASnet }

     Peter,

     How one addresses BITNET through the DASnet Service depends on the
     source system.  From GeoMail, it's as follows:

     To use DASnet(R) to send me electronic mail from GeoMail, send mail
     as follows:

{\tx To: GEO4:DASNET }\nl
{\tx Subject: user@site.bitnet!the subject }

     One could address to anyone on the Internet in the same way.

     [DASnet are on +1 408 559 7434]

\bigskip
\hrule%-----------------------------------------------------------------
\bigskip

     This document is an abbreviated version of a chapter on \TeX\
     servers in {\it The \TeX\ Companion} by Adrian Clark (in
     perparation). All the data on servers is present, but a substantial
     amount of tutorial material on networking has been omitted here.
     If you want the full works, buy the book when it is published!

     As always, I must thank the many contributors to this collection.
     It is usually invidious to name names, but special thanks must go
     to Don Hosek for his list of electronic publications; Peter Abbott,
     Adrian Clark and Brian Hamilton Kelly for their work on the Aston
     archive; Carl Witty for helping me with access to the FTP servers;
     Max Hailpern for the information on the \LaTeX\ volunteer corps;
     James van Zandt for the list of Internet sites with \TeX-related
     material, and all those who pointed me in the direction of material
     I had not come across before.
\bye

